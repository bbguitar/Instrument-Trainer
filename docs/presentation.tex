\documentclass[english]{beamer}

\usepackage{verbatim}
\usepackage{listings}
\usepackage{caption}

\setbeamercovered{transparent}

\DeclareCaptionFont{white}{\color{white}}
\DeclareCaptionFormat{listing}{\colorbox{gray}{\parbox{\textwidth}{#1#2#3}}}
\captionsetup[lstlisting]{format=listing,labelfont=white,textfont=white}

\lstset{
language=C,
basicstyle=\small\sffamily,
%numbers=left,
numberstyle=\tiny,
frame=tb,
columns=fullflexible,
showstringspaces=false
}

\usetheme{Warsaw}
%\usetheme[height=7mm]{Rochester}

\AtBeginSection[]
{
  \begin{frame}
    \frametitle{Outline}
    \tableofcontents[currentsection]
  \end{frame}
}

\title[Training software for learning to play musical instruments]{Instrument Trainer}
\author[J. Elser, C. Fackler]{Josh Elser, Cameron Fackler}
\institute{
Rensselaer Polytechnic Institute\\
ITEC 4961: IT for Arts and Performance\\
}
\date[5/6/2010]

\begin{document}

\begin{frame}
  \frametitle{Final Project}
  \titlepage
\end{frame}

\begin{frame}
  \frametitle{Outline}
  \tableofcontents
\end{frame}

\section{Introduction}

\subsection{What is it?}

\begin{frame}
  \frametitle{Instrument Trainer}
  \begin{itemize}
    \item \uncover<1->{Instrument Trainer is a C++ application written with \textit{QT} that
      compares a stream of incoming MIDI data to a MIDI file, displaying the
      accuracy of the notes played by the user.}
    \item \uncover<2->{By converting data taken from a microphone into MIDI
      notes, Instrument Trainer can follow along with a played melody.}
  \end{itemize}
\end{frame}

\section{Technology}

\subsection{MIDI}

\begin{frame}
  \frametitle{MIDI}
  \begin{itemize}
    \item \uncover<1->{MIDI provides a simple way to compare notes in the
      application.}
    \item \uncover<2->{Many channels and fast data transfer is not necessary.}
  \end{itemize}
\end{frame}

\subsection{ALSA - MIDI Sequencer}

\begin{frame}
  \frametitle{ALSA - MIDI Sequencer}
  \begin{itemize}
    \item \uncover<1->{Need to have a way to pass MIDI data from the pitch
      detection to the application.}
    \item \uncover<2->{Creates a "pipe" to send MIDI data over.}
  \end{itemize}
\end{frame}

\subsection{JACK}

\begin{frame}
  \frametitle{Taco}
  Taco taco taco
\end{frame}

\subsection{DSP - Aubio}

\begin{frame}
  \frametitle{Taco}
  Taco taco taco
\end{frame}

\subsection{OpenGL}

\begin{frame}
  \frametitle{Taco}
  Taco taco taco
\end{frame}

\section{What was accomplished}

\begin{frame}
  \frametitle{Taco}
  Taco taco taco
\end{frame}

\section{Potential Issues}

\begin{frame}
  \frametitle{Taco}
  Taco taco taco
\end{frame}

\section{Future Work}

\begin{frame}
  \frametitle{Taco}
  Taco taco taco
\end{frame}

\end{document}
