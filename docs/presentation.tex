\documentclass[english]{beamer}

\usepackage{verbatim}
\usepackage{listings}
\usepackage{caption}

\setbeamercovered{transparent}

\DeclareCaptionFont{white}{\color{white}}
\DeclareCaptionFormat{listing}{\colorbox{gray}{\parbox{\textwidth}{#1#2#3}}}
\captionsetup[lstlisting]{format=listing,labelfont=white,textfont=white}

\lstset{
language=C,
basicstyle=\small\sffamily,
%numbers=left,
numberstyle=\tiny,
frame=tb,
columns=fullflexible,
showstringspaces=false
}

\usetheme{Warsaw}
%\usetheme[height=7mm]{Rochester}

\AtBeginSection[]
{
  \begin{frame}
    \frametitle{Outline}
    \tableofcontents[currentsection]
  \end{frame}
}

\title[Training software for learning to play musical instruments]{Instrument Trainer}
\author[J. Elser, C. Fackler]{Josh Elser, Cameron Fackler}
\institute{
Rensselaer Polytechnic Institute\\
ITEC 4961: IT for Arts and Performance\\
}
\date[5/6/2010]

\begin{document}

\begin{frame}
  \frametitle{Final Project}
  \titlepage
\end{frame}

\begin{frame}
  \frametitle{Outline}
  \tableofcontents
\end{frame}

\section{Introduction}

\subsection{What is it?}

\begin{frame}
  \frametitle{Instrument Trainer}
  \begin{itemize}
    \item \uncover<1->{Instrument Trainer is a C++ application written with \textit{QT} that
      compares a stream of incoming MIDI data to a MIDI file, displaying the
      accuracy of the notes played by the user.}
    \item \uncover<2->{By converting data taken from a microphone into MIDI
      notes, Instrument Trainer can follow along with a played melody.}
  \end{itemize}
\end{frame}

\section{Technology}

\subsection{MIDI}

\begin{frame}
  \frametitle{MIDI}
  \begin{itemize}
    \item \uncover<1->{MIDI provides a simple way to compare notes in the
      application.}
    \item \uncover<2->{Many channels and fast data transfer is not necessary.}
  \end{itemize}
\end{frame}

\subsection{ALSA - MIDI Sequencer}

\begin{frame}
  \frametitle{ALSA - MIDI Sequencer}
  \begin{itemize}
    \item \uncover<1->{Need to have a way to pass MIDI data from the pitch
      detection to the application.}
    \item \uncover<2->{Creates a ``pipe'' to send MIDI data over.}
  \end{itemize}
\end{frame}

\subsection{JACK}

\begin{frame}
  \frametitle{JACK}
  \begin{itemize}
    \item \uncover<1->{JACK Audio Connection Kit}
    \item \uncover<2->{``System for handling real-time, low-latency audio and
      MIDI''}
    \item \uncover<3->{Allows us to process microphone data in real-time}
  \end{itemize}
\end{frame}

\subsection{DSP - Aubio}

\begin{frame}
  \frametitle{Aubio}
  \begin{itemize}
    \item \uncover<1->{``Aubio is a tool designed for the extraction of audio
      annotations from audio signals''}
    \item \uncover<2->{Allows us to generate MIDI events from microphone data}
    \item \uncover<3->{Uses JACK to process the microphone data in real-time}
  \end{itemize}
\end{frame}

\subsection{OpenGL}

\begin{frame}
  \frametitle{OpenGL}
  \begin{itemize}
    \item \uncover<1->{``OpenGL is the premier environment for developing
      portable, interactive 2D and 3D graphics applications.''}
    \item \uncover<2->{Allows the user to play along with a song, the notes
      need to be rendered to the screen.}
    \item \uncover<3->{Can draw a staff, key signature, time signature, notes,
        and a progress bar while playing.}
  \end{itemize}
\end{frame}

\section{What was accomplished}

\begin{frame}
  \frametitle{Accomplishments}
  \begin{itemize}
    \item \uncover<1->{Listen to a played instrument, converting audio signal to
      a pitch, then to a MIDI event.}
    \item \uncover<2->{Render song data stored in a MIDI file.}
    \item \uncover<3->{Compare stored data to incoming MIDI data (a.k.a. compare
      the song to the instrument performance).}
    \item \uncover<4->{User is able to select treble or bass clef for the song
      to be displayed as.}
    \item \uncover<5->{User is able to transpose the instrument input to allow
      for non-concert-pitch instruments or microphone inconsistencies.}
  \end{itemize}
\end{frame}

\section{Potential Issues}

\begin{frame}
  \frametitle{Potential Issues}
  \begin{itemize}
    \item \uncover<1->{Lacking microphone quality.}
    \item \uncover<2->{Interference from outside noise.}
    \item \uncover<3->{Slow OpenGL rendering with a large window.}
  \end{itemize}
\end{frame}

\section{Future Work}
%- Continue cleaning up the UI
%- make graphics rendering more efficient

\begin{frame}
  \frametitle{Future Work}
  \begin{itemize}
    \item \uncover<1->{Continue to optimize user interface.}
    \item \uncover<2->{Increase efficiency of graphic rendering.}
    \item \uncover<3->{More robust musical notation.}
  \end{itemize}
\end{frame}

\section{References}

\begin{frame}
  \frametitle{References}
  \begin{itemize}
    \item \url{http://github.com/joshelser/Instrument-Trainer/}
    \item \url{http://www.alsa-project.org/main/index.php/Main_Page}
    \item \url{http://jackaudio.org/}
    \item \url{http://aubio.org/}
    \item \url{http://www.opengl.org/}
  \end{itemize}
\end{frame}

\end{document}
